\section{Einleitung}
Im Zuge der "`digitalen Revolution"' (Fußnote), die durch das mooresche Gesetz(Fußnote) beschrieben wird und durch die fortschreitende Miniaturisierung, können immer komplexere und vielseitigere Aufgaben von Kleinstrechnern bearbeitet werden. Diese ermöglichen es uns eine Logik und Kombinatorik in kleinen, mobilen Geräten aufzubauen, die Menschen in ihrem Verhalten ähnelt und im Alltäglichen Verwendung finden kann. Dies können wir wiederum nutzen, um unsere Aufgaben auf die Maschinen zu verteilen und den Menschen zu entlasten. Eine Aufgabe ist dabei das Führen eines Fahrzeuges ohne menschliche Eingabe. Das autonome Fahren ist hierbei keine Fiktion mehr, wie in Filmen wie iRobot (Fußnote), oder Demolition Man(Fußnote) sugeriert wird, sondern der Prozess ist schon soweit fortgeschritten, dass er in unserer Gesellschaft angekommen ist. Dafür spricht zum einen, dass schon die ersten autonomen Fahrzeuge des "`EUREKA-PROMETHEUS-Projekts"' (Fußnote) vor gut 20 Jahren weit mehr als 1758 Km auf öffentlichen Straßen zurück gelegt haben. Und zum anderen, dass in den USA und Europa erste autonome Fahrzeuge für den Straßenverkehr zugelassen(Fußnote-das mit Nevada und der A9) und neue Gesetze dafür entworfen werden(Fußnote-das mit Dobrindt). Somit stellen sich nur noch die Fragen, wann der Straßenverkehr auf autonome Fahrzeuge umgestellt und wie es am Ende realisiert wird.\\ \\
Vor diesem Hintergrund findet der Carolo-Cup in Braunschweig seit nun mehr acht Jahren statt, welcher studentische Teams aller Fachrichtungen und Universitäten in einem Wettbewerb gegeneinander antreten lässt, um das beste Konzept und die beste Umsetzung eines autonomen Fahrzeuges zu präsentieren. Die Otto-von-Guericke-Universität Magdeburg beteiligt sich ebenfalls an diesem Wettbewerb mit dem Projekt "`oTToCAR"'. Die einzelnen Disziplinen sind Einpark,- Spurverfolgungs- und Fahrspurwechselszenarien, wofür neben der Hardware- und Software-Entwicklung auch ein Reglungskonzept entwickelt werden muss. Die vorliegenen Arbeit beschäftigt sich mit einer modellbasierten Regelung für alle Szenarien und das dafür benötigte Modell mit der nötigen Parameterschätzung. In einer vorrangegangenen Arbeit \cite{VikAnd} wurde bereits ein Modell entwickelt, deren Parameter jedoch augrund eines fehlenden realen Fahrzeuges nicht bestimmt werden konnten. Die jetzige Arbeit ist zeitlich später einzuordnen, in der ein fertiger Prototyp bereits zur Verfügung stand und die Parameterschätzung vollendet werden konnte.