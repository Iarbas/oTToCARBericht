%%%%%%%%%%%%%%%%%%%%%%%%%%%%%%%%%%%%%%% Titelseite
\titlehead{%
\begin{footnotesize}
   \parbox[b][][c] {6.1cm} { %
      \includegraphics[width=4.5cm]{Bilder/STAC.png}}
   \parbox{0.5cm} {}
   \parbox[b][2.0cm][c] {6cm} { %
      \includegraphics[width=4.5cm]{Bilder/Otto.png}}
   \parbox{0.5cm} {}
   \parbox[b][2.0cm][c] {4.5cm} { %
      \includegraphics[width=4.5cm]{Bilder/Ottocar.png}}
\end{footnotesize}
\vspace{10mm}
}

\subject{Interdisziplinäres Teamprojekt}

\title{Parameterschätzung und\\ Entwurf einer Modellprädiktiven Regelung \\ eines autonomen Modellfahrzeuges\\
\vspace{5mm}
\textnormal{\small von}}

\author{
\textnormal{\large Hannes Heinemann}\\[-3mm] \textnormal{\large Matrikelnummer: 184102} \and 
\textnormal{\large André Pieper}\\[-3mm] \textnormal{\large Matrikelnummer: 184960} \\
\vspace{1mm}
}

\date{2. April 2015}

\publishers{%
 \vspace{1em}
 \begin{normalsize}
   \centering
   {\large Betreuer:\\}
   \medskip
   M. Sc. Juan Pablo Zometa
   \thanks{Otto-von-Guericke-Universität Magdeburg, Fakultät für Elektro- und Informationstechnik, Institut für Automatisierungstechnik - IFAT, Lehrstuhl für Systemtheorie und Regelungstechnik, Prof. Dr.-Ing. Rolf Findeisen, Universitätsplatz 2, 39106 Magdeburg} ,\\
   M. Sc. Michael Maiworm $^*$
 \end{normalsize}
 \vspace{1mm}
}

\begin{document}

\maketitle 
\setcounter{tocdepth}{2}

\section*{Kurzdarstellung}
In diesem Bericht werden Werkzeuge und Verfahren aufgezeigt, um für ein autonomes Modellfahrzeug im Maßstab 1:10 eine Fahrspurverfolgung zu entwickeln. Diese Arbeit ist Teil des studentischen Teamprojekts "`oTToCar"' der OvGU und wurde unter der Berücksichtigung der Teilnahme an dem internationalen Wettbewerbs "`Carolo-Cup"' \cite{RegelW} entworfen. Die Betreuung und Zusamenarbeit erfolgt dabei durch die Fakultäten für Informatik, für Elektrotechnik und Informationstechnik und für Maschinenbau. Die Vorbereitung dieser Arbeit ist maßgeblich an das Forschungsprojekt \cite{VikAnd} von Viktoria Wiedmeyer und Andreas Himmel geknüpft, auf das sich das im folgenden verwendete Modell der Fahrzeugdynamik hauptsächlich bezieht. 