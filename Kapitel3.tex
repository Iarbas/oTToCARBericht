\begin{comment}
\section{Modellprädiktive Regelung}
Allgemeines zu MPC
\\
\subsection{Grundlagen der modellprädiktiven Regelung}
Systemgleichung oft in zeitdiskreter Form, da Optimierung in kontinuierlicher Form meist komplexer
\begin{align}
  Systemgleichungen
\end{align}
Ausgehend von $x(k)$ dem aktuellen Zustand des Systems, der wenn nicht messbar geschätzt werden muss, wird anhand des Systemmodells das zukünftige Systeverhalten
\begin{align}
  Sequenz von Zuständen
\end{align}
bis zum Prädiktionshorizont $n_p$ unter der Optimierung einer Sequenz von Eingängen
\begin{align}
  Sequenz von Eingängen
\end{align}
bis zum Stellhorizont $n_c$ vorhergesagt. Aus der gefundenen optimalen Eingangssequenz $u*$ wird der erste Eintrag $u*(k)$ auf das zu regelnde System angewandt.\\
Nach der Zeit $\delta t$ kann der neue Zustand gemessen werden und die Optimierung beginnt von neuem.\\
Kostenfunktion meist quadratisch.
\begin{align}
  quadratische Kostenfunktion
\end{align}
Linear MPC/Nonlinear MPC\\
Beschränkungen
\subsection{Konkrete Umsetzung der modellprädiktiven Regelung am oTToCAR}

Vereinfachungen\\
Stabilität\\
Skalierbarkeit\\
online Tuning mit dynamic reconfigure (welche Parameter)\\
\subsection{Zukünftige Schritte}
\end{comment}