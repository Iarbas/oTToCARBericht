\documentclass[10pt,a4paper,oneside,headinclude,footinclude,BCOR5mm]{scrartcl}
\usepackage{pdflscape}                         % fuer Querformate und so
\usepackage{wrapfig}                           % ermoeglicht, Bilder von Text umfliessen zu lassen
\usepackage[pdftex]{graphicx}                  % einbinden Graphiken
\usepackage[utf8]{inputenc}                    % für die Kodierungsangabe des Editors
\usepackage[T1]{fontenc}                       % Deutschzeug
%\usepackage[marvosym]                         %
\usepackage{ngerman}                           % 
\usepackage{mathrsfs,amssymb}                  % Mathezeug
\usepackage[intlimits]{empheq}                 %
\usepackage{amsmath}                           %
\usepackage{amsfonts}                          %
\usepackage{array}                             % Tabellenerweiterung
\usepackage{tabularx}
\usepackage{ifthen,calc}                       % Verbesserte Zuweisungen und Kontrollstrukturen
\usepackage{pifont,charter,courier}            % Schriften und Symbole
\usepackage[scaled]{helvet}                    %
%\usepackage{natbib}
\usepackage[numbers,square]{natbib}            % fuer den Bibtex, damit plaindin verwendet werden kann
\usepackage{caption}                           % fuer die Beschriftung von Grafiken
\usepackage{float}                             % macht die figures dorthin, wohin ich sie haben moechte
\usepackage{hyperref}
\usepackage{breakurl}

% Zur Veraenderung der Ueberschriften
\usepackage{setspace}                          % fuer den Zeilenabstand
\usepackage{titlesec}                          % zur Formatierung der Ueberschriften
\titleformat{\chapter}[hang]{\normalfont\Large\bfseries}{\thechapter}{0.5em}{}          % Format muss beim chapter mit rein, sonst geht es nicht.
\titleformat{\section}[hang]{\normalfont\large\bfseries}{\thesection}{0.5em}{}
\titleformat{\subsection}[hang]{\normalfont\bfseries}{\thesubsection}{0.5em}{}

\titlespacing*{\chapter}
{0pc}{-20pt}{6pt}[0pc]

\titlespacing{\section}
{0pc}{10pt}{*1}[0pc]
\titlespacing{\subsection}
{0pc}{6pt}{*1}[0pc] 

\usepackage{accents}                           % siehe eine Zeile drunter
\renewcommand{\dot}[1]{%                       % fuer einen besseren Punkt bei Zeitableitungen
  \accentset{\mbox{\large .}}{#1}}
\renewcommand{\ddot}[1]{%
  \accentset{\mbox{\large .\hspace{-0.2ex}.}}{#1}}

% Zeug fuer die PGF-Plots: http://wordpress.alphao.org/2013/07/26/how-to-plot-matlab-figures-with-latex-using-pgfplots-and-csv-formatted-data-file/
\usepackage[margin=0em]{geometry}
\usepackage{pgfplots}
\usepackage{pgfplotstable}
%\pgfplotsset{compat=1.8}

% Einstellungen
\usepackage{parskip}                                 % Einzug aus, Absatzabstand ein
\usepackage{pdfpages}                                % ermoeglicht pdf-Seiten einzufuegen
\usepackage{ulem}                                    % neue Strich-Funktionen (z.B. Text durchstreichen)
\tolerance=2000                                      % Toleranz für Wortzwischenräume
\setlength{\emergencystretch}{20pt}                  % Zusätliche Zeilendehnbarkeit
\setlength{\fboxrule}{1pt}\setlength{\fboxsep}{4mm}  % Rechtecke um Gleichungen

\setlength\voffset{-1in}                             % Hier drunter ist das Zeug von Andreas
\setlength\hoffset{-1in}
\setlength\topmargin{2cm}
\setlength\oddsidemargin{2cm}
\setlength\textheight{23.7cm}
\setlength\textwidth{17.001cm}
\setlength\footskip{1cm}
\setlength\headheight{0.6cm}
\setlength\headsep{1cm}
\setlength{\skip\footins}{0.119cm}
\setlength\tabcolsep{1mm}
\renewcommand\arraystretch{1.3}
\renewcommand{\footnoterule}{\rule{200pt}{1pt}{\vspace*{0.5cm}}}

\renewcommand{\figurename}{Abb.} 				% Abkuerzung von Abbildung

\usepackage{fancyhdr}                          % Einstellungen für Kopf- und Fusszeilen
\pagestyle{fancy}
\fancyhf{}
\fancyhead[RO,LE]{\thepage}                    % ungerade rechts, gerade links
\fancyhead[C]{\nouppercase{\leftmark}}
\usepackage{verbatim}    %Macht unter anderem Blockkommentare (\begin{comment) *Kommentar* \end{comment} möglich