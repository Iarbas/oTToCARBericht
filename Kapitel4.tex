\section{Fahrspurregelung mittels modellprädiktiver Regelung}
In allen dynamischen Disziplienen des Carolo-Cups besteht die wichtige Teilaufgabe möglichst fehlerfrei einer in weiß auf schwarz aufgeklebten Fahrbahn zu folgen. Um diese Aufgabe zu erfüllen, wurde ein modellprädiktiver Regelansatz gewählt.\\ \\ 
Bei der modellprädiktiven Regelung (MPC) handelt es sich um eine Form der optimalen Regelung, bei der wiederholt eine Berechnung der optimalen Steuerung für ein System ausgehend von dessen aktuellem Zustand stattfindet. In vielen Bereichen finden MPCs immer häufiger Anwendung, da sie eine direkte Berücksichtigung von Beschränkungen erlauben und eine Form des strukturierten Reglerentwurfs ausgehend von der modellierten Systemdynamik darstellen. Dabei kann durch die geeignete Wahl der Kostenfunktion und deren Wichtungsparametern die Güte des Reglers geziehlt beeinflusst werden. Allerdings ergeben sich auch Schwierigkeiten bei der Verwendung von MPCs. Zum einen ist die Konvergenz der Optimierung gegen einen optimalen Wert für die Optimierungsvariablen und die Stabilität des geschlossenen Kreises insbesondere bei nichtlinearen Systemmodellen oft nur schwierig nachweisbar und zum anderen stellt das wiederholte Lösen des meist hochdimensionalen Optimierungsproblems während der Laufzeit in genügend schneller Geschwindigkeit eine große Herausforderung dar.
\\
\subsection{Problemformulierung für die modellprädiktive Regelung}
Im realen Anwendungsfall des oTToCAR-Projekts eignet sich eine Systemdarstellung in zeitdiskreter Form (\cite{ADA09}), bei der die Lösung des Optimierungsproblems weniger komplex ist und die ebenfalls zeitdiskreten Messwerte vom realen System weniger kompliziert integriert werden können. Demnach sind die diskretisierten Systemgleichungen wie folgt gegeben:
\begin{align*}
  \boldsymbol{x}(k+1)&=\boldsymbol{f}\left ( \boldsymbol{x}(k), \boldsymbol{u}(k) \right )\\
  \boldsymbol{y}(k)&=\boldsymbol{g}\left ( \boldsymbol{x}(k) \right )\\
\end{align*}
mit den nichtlinearen Funktionen $\boldsymbol{f}\left ( \cdot \right )$ und $\boldsymbol{g}\left ( \cdot \right )$, wobei
\begin{align*}
  &\boldsymbol{x}(k) \in \mathcal{X}\subset\mathbb{R}^n\\
  &\boldsymbol{u}(k) \in \mathcal{U}\subset\mathbb{R}^m\\
  &\boldsymbol{y}(k) \in \mathcal{Y}\subset\mathbb{R}^r
\end{align*}
Ausgehend vom aktuellen Zustand $\boldsymbol{x}(k)$ des zu regelnden Systems, der wenn nicht messbar geschätzt werden muss, wird anhand des Systemmodells das zukünftige Systemverhalten
\begin{align*}
  \boldsymbol{x_p}=\left\{ \boldsymbol{x}(k+1),\dots,\boldsymbol{x}(k+n_p)\right\}
\end{align*}
bis zum Prädiktionshorizont $n_p$ unter der Optimierung einer Sequenz von Eingängen
\begin{align*}
  \boldsymbol{u}=\left\{ \boldsymbol{u}(k),\dots,\boldsymbol{u}(k+n_c-1)\right\}
\end{align*}
bis zum Stellhorizont $n_c$ vorhergesagt. Aus der gefundenen optimalen Eingangssequenz $\boldsymbol{u}^*$ wird der erste Eintrag $\boldsymbol{u}^*(k)$ auf das zu regelnde System angewandt. Im nächsten Zeitschritt kann der neue Zustand gemessen bzw. geschätzt werden und die Optimierung beginnt von neuem. Ziel dabei ist es einer Referenztrajektorie $\boldsymbol{x_r}$ zu folgen.\\ \\
Für das an jedem Zeitschritt $k$ zu lösende Minimierungsproblem wurde die benötigte Kostenfunktion $J$ in quadratische Form mit $\boldsymbol{x_p}$ und $\boldsymbol{u}$ als Optimierungsvariablen aufgestellt:
\begin{align*}
	\underset{\boldsymbol{x_p, u}}{\text{min}}\;&J:=\sum_{i=k+1}^{k+n_p} \left [\boldsymbol{x}_{p}(i)-\boldsymbol{x}_{r}(i)\right ]^T\boldsymbol{Q}_i\left [\boldsymbol{x}_{p}(i)-\boldsymbol{x}_{r}(i)\right ] +\sum_{j=k}^{k+n_c-1} \boldsymbol{u}^T(j)\boldsymbol{R}_j\boldsymbol{u}(j)\\
	s.t.\;&\boldsymbol{x_p}(i+1)=\boldsymbol{f}\left ( \boldsymbol{x_p}(i), \boldsymbol{u}(i) \right ),\quad i=k,...,k+n_c-1\\
	&\boldsymbol{x_p}(i+1)=\boldsymbol{f}\left ( \boldsymbol{x_p}(i), \boldsymbol{u}(k+n_c-1) \right ),\quad i=k+n_c,...,k+n_p-1
\end{align*}
Mit den Vektoren
\begin{align*}
	\boldsymbol{x}_p(k)&=\left [ \boldsymbol{x}_p(k+1\mid k),\dots,\boldsymbol{x}_p(k+n_p\mid k) \right ]^T\\
	\boldsymbol{x}_r(k)&=\left [ \boldsymbol{x}_r(k+1),\dots,\boldsymbol{x}_r(k+n_p) \right ]^T\\
	\boldsymbol{u}(k)&=\left [ \boldsymbol{u}(k),\dots,\boldsymbol{u}(k+n_c-1) \right ]^T
\end{align*}
und den dazugehörigen positiv definiten Wichtungsmatrizen $\boldsymbol{Q}_i\in\mathbb{R}^{n\times n}\;(i=1, ...,n_p)$ und $\boldsymbol{R}_j\in\mathbb{R}^{m\times m}\;(j=0, ...,n_c-1)$. Weiterhin lässt sich das Optimierungsproblem um einfache Beschränkungen der Eingänge
\begin{align*}
  \boldsymbol{u}_{min} \leq \boldsymbol{u}(i) \leq \boldsymbol{u}_{max},\quad i=k,...,k+n_c-1
\end{align*}
und Zustandsbeschränkungen der Form
\begin{align*}
  \boldsymbol{A}\boldsymbol{x}_p(i) \leq \boldsymbol{b}\quad i=k+1,...,k+n_p
\end{align*}
erweitern.

\subsection{Implementierung der Fahrspurregelung für das oTToCAR}
Um die Theorie in die Praxis umzusetzen, muss der Algorithmus der MPC um die Erstellung einer Referenztrajektorie erweitert, sowie Vereinfachungen am Algorithmus vorgenommen werden. Denn der Schwachpunkt einer jeder modellprädiktiven Regelung ist die Laufzeit des Programms bei einer Online-Berechnung der nächsten Stellgröße. Somit muss ein Kompromiss zwischen Güte der Stellgröße und der Zeit zur Berechnung für diese gefunden werden.
\subsubsection{Dimension der Optimierungsvariable}
Als Systemeingänge für das Fahrzeugmodell sind der Lenkeinschlag der Räder $u_1=\delta$ und das Motordrehmoment $u_2=M_A$ vorhanden, demnach ergibt sich der Stellgrößenvektor zum Zeitpunkt $k$ zu $\boldsymbol{u}(k)=[u_1(k), u_2(k)]^T$. Da sich die prädizierten Zustände $\boldsymbol{x}_p$ aus den Gleichungsnebenbedingungen berechnen lassen, ergibt sich die Dimension der Optimierungsvariable also zu $2\cdot n_c$.
Je größer man $n_c$ wählt, desto besser lässt sich z.B. die Geschwindigkeit in Abhängigkeit zur Entfernung und Krümmung einer bevorstehenden Kurve begrenzen oder Schwierigkeiten beim Durchfahren von S-Kurven bewältigen. Allerdings nimmt dadurch auch die Dimension der Optimierungsvariable zu, was dazu führt, dass die Grenzen der verfügbaren Rechenzeit zum Lösen des Optimierungsproblems während der Laufzeit schnell erreicht werden.
\subsubsection{Referenztrajektorie}
Mit einer auf dem Fahrzeug befestigten Kamera lässt sich die zu verfolgende Fahrbahn wahrnehmen. Diese Wahrnehmung liefert mit einer Rate von bis zu 50Hz eine Polylinie der erkannten Fahrbahnmarkierung projiziert auf die rechte Fahrspur. Diese Polyline kann je nach erkannter Fahrbahn aus einer unterschiedlich Anzahl von Punkten mit variablem Abstand zu einander bestehen.\\ \\
Um diese Polylinien als Referenztrajektorie für die MPC nutzen zu können, werden zu kurze Polylinien zunächst soweit extrapoliert, dass ein Vergleich zwischen Prädiktion zum Zeitschritt $k+np$ und Referenz in jedem Fall möglich ist. Anschließend werden die Koordinaten der Punkte auf der Polylinie als Funkion in Abhängigkeit der summierten Wegstrecke zwischen den Punkten (vom Auto ausgehend) hinterlegt, um diese als Referenztrajektorie zu den dazugehörigen Werten aus der Prädiktion mit der zurückgelegten Wegstrecke $s_i = iv\Delta t$ interpolieren zu können. Dabei ist $v$ die Geschwindigkeit der Fahrzeuges. In Abb. \ref{fig:referenz} ist ein Beispiel einer Referenztrajektorie dargestellt.
\begin{figure}[t]
\centering
\includegraphics[scale=0.53]{Bilder/Reference.png}
\caption{Veranschaulichung der Referenztrajektorie und des prädizierten Verhaltens aufgetragen im Weltkoordinatensystem $x_1$ über $x_2$ in Meter}
\label{fig:referenz}
\end{figure}
Die Terme $\left [\boldsymbol{x}_{p}(i)-\boldsymbol{x}_{r}(i)\right ]$ in der Kostenfunktion werden demnach durch $\left [\boldsymbol{x}_{p}(i)-\boldsymbol{x}_{r}(s_i)\right ]$ ersetzt.\\ \\
Das Modell liefert als $x_1$ und $x_2$ die globale Position des Autos und den dazugehörigen Gierwinkel $x_3=\psi$. Da bisher noch keine globale Karte implementiert ist, wird jedes Mal, wenn eine neue Polylinie übermittelt wird die jeweils lokal gültige Weltkarte auf das Fahrzeugkoordinatensystem zurück gesetzt (Fahrzeugposition und Gierwinkel werden auf 0 zurück setzen.) Es ist kurzzeitig möglich, bei Ausbleiben einer verlässlichen neuen Polylinie, weiter an der letzten Referenztrajektorie weiter zu fahren, in dem man den Zustand bezogen auf das letzte lokal gültige Weltkoordinatensystem mit Hilfe des Modells schätzt. Dies ist allerdings nur für sehr kurze Zeiten (einzelne ausbleibende Polylinien) verlässlich.
\subsubsection{Vereinfachungen}
Aufgrund zwischenzeitlich erschöpften Rechenkapazitäten auf dem Fahrzeug während der Entwicklungsphase wurde der modellprädiktive Ansatz zunächst auf die einfachste denkbare Implementierung vereinfacht. Dabei wurde darauf verzichtet die Motorsteuergröße mit zu optimieren und der Stellhorizont für den Lenkeinschlag auf $n_c=1$ begrenzt. Aufgrund überraschend guter Ergebnisse wurde an dieser Vereinfachung auch im Carolo-Cup festgehalten.\\ \\
Im nun eindimensionalen Optimierungsproblem lässt sich der Verlauf der Kostenfunktion in Abhängigkeit vom Lenkeinschlag sehr anschaulich darstellen. Schnell wird dabei klar, dass die Kostenfunktion im Bereich der real möglichen Lenkeinschläge eine Parabelform aufweist. Der Scheitelpunkt der Parabel und damit eine sehr gute Approximation des optimalen Lenkeinschlags lässt sich mit nur 3 Funktionsaufrufen bestimmten. Dieses Vorgehen führt zu imensen Rechenzeiteinsparungen.\\ \\
Da die Stellgröße für das Motordrehmoment nicht optimiert wird, muss die Anpassung der Geschwindigkeit auf andere Weise vorgenommen werden, wenn das Fahrzeug auf den langen Geraden möglichst schnell fahren soll. Die Vorgabe der Geschwindigkeit erfolgt dabei aus einer einfachen Abhängigkeit von der mittleren prädizierten Gierrate $\dot{\psi}$ und wird von einem unterlagertem Geschwindigkeitsregler sichergestellt.
\subsubsection{Stabilität}
Um die Garantie zu haben, dass der Algorithmus immer gegen ein Optimum konvergiert, wurde die im implementierten Fall skalare Kostenfunktionen während Fahrt auf dem Parcours in möglichst vielen denkbaren Positionen aufgenommen und überprüft, ob sich Fälle ergeben in denen die Kostenfunktion im relevanten Bereich der möglichen Lenkeinschläge nicht convex ist. Abb. \ref{fig:parabel} zeigt das Ergebnis dieser Untersuchung. Es ist zu erkennen, dass die Kostenfunktion nur in wenigen Ausnahmefällen einer nach unten geöffneten Parabel ähnelt. Der hierbei berechnete Lenkeinschlag wird nicht auf das System gegeben, stattdessen wird an der vorherigen berechneten Referenztrajektorie weiter gefahren und auf die nächste Polylinie von der Wahrnehmung gewartet. Dank der relativ hohen Wiederholrate des Reglers stellen diese vereinzelten Ausfälle kein Problem dar.
\begin{figure}[t]
\centering
\includegraphics[scale=0.75]{Bilder/Parabeln.png}
\caption{Kostenfunktion in Abhängigkeit vom Radeinschlag in verschiedenen Fahrzeugpositionen}
\label{fig:parabel}
\end{figure}
\subsection{Spurwechsel}
In einer dynmaischen Disziplien des Carolo-Cups ist es erforderlich Hindernissen auf der Fahrbahn auszuweichen. Dazu wird die erkannte Polylinie auf die linke Farspur verschoben, sobald ein Hindernis detektiert wird, wie in Abb. \ref{fig:spurwechsel} dargestellt. Die MPC soll die optimale Trajektorie zu den verschobenen Punkten finden, sodass keine gesonderte Fallbetrachung mit alternativen Reglreinstellungen vorgenommen werden muss. Ein Problem stellt dabei allerdings die Wahrnehmung dar, da bei zu starken Einlenken die Fahrbahn aus dem Sichtfeld der Kamera verschwindet. Alternativ eine langsamere Spurwechseltrajektorie vorzugeben bei der die Fahrbahn weiterhin wahrgenommen wird, ist nicht denkbar, da Hindernisse nach einer Kurve oft erst spät erkannt werden und deshalb Spurwechsel auf möglichst kürzester Distanz nötig sind. Da das Verfolgen der zuletzt gesehenen Polylinie wegen Modellungenauigkeiten und begrenzter Länge der Polylinie nur kurzzeitig ausreichend gut funktioniert, schlägt die implementierte Strategie immernoch häufig fehl. Dazu wird allerdings wie schon oben erwähnt eine wirkliche globale Karte implementiert, in der die komplette Referenztrajektorie eingetragen ist und die tatsächliche Position des Fahrzeugs im Weltkoordinatensystem anhand der erkannten Strecke besser geschätzt werden kann.
%\begin{wrapfigure}{r}{8cm} 
\begin{figure}[t]
\centering
\includegraphics[scale=0.55]{Bilder/Spurwechsel.png}
\caption{Verschieben der Referenztrajektorie auf die linke Fahrspur bei erkanntem Hindernis}
\label{fig:spurwechsel}
\end{figure}
%\end{wrapfigure}
\subsection{Dynamic Reconfigure}
In der unmittelbaren Vorbereitung auf den Wettkampf hat sich herausgestellt, dass die kurzen Testzeiten auf der Originalstrecke gut ausgenutzt werden müssen. Deswegen musste bereits im Vorfeld Überlegungen angestellt werden, welche Parameter entscheidenen Einfluss auf die Güte der Regelung haben, um diese geziehlt zu beeinflussen. Zur Einstellung der Parameter liefert das genutze Kommunikations Framework ROS mit dem Dynamic Reconfigure Paket die ideale Funktion. So kann online während der Fahrt auf dem Parcours auf die wichtigsten Parameter wie die maximale Geschwindigkeit $v_{max}$, der Koeffizient zur Drosslung der Geschwindigkeit in den Kurven sowie die Einträge der Wichtungsmatrizen Einfluss genommen werden. Der direkte Eindruck von Ändrungen im Verhalten des Fahrzeuges auf der Strecke erleichtert das tunen des Reglers erheblich.
\subsection{Zukünftig geplante Verbesserungen des Algorithmus}
Im nächsten Schritt der Entwicklung sollen sofern dies möglich ist nach und nach die angewandten Vereinfachungen rückgängig gemacht werden. So verspricht die Erhöhung des Stellhorizontes $n_c$ genauere Informationen von der Prädiktion über die zukünftige Bewegung des Fahrzeuges. Damit sollte es möglich sein schon eher vor Kurven zu bremsen und ebenfalls frühzeitiger aus Kurven heraus beschleunigen zu können, um eine im Mittel eine höhere Geschwindikeit zu erreichen. Auch die Mitoptimierung der Geschwindigkeit sollte eine geziehltere Anpassung dieser begünstigen.\\ \\ Andere Teile des oTToCAR-Teams beschäftigen sich außerdem mit der Erstellung einer globalen Karte zur Vorgabe als Referenztrajektorie und Möglichkeit einer genauerer Schätzung des aktuellen Zustands. Diese macht weitere bereits angedeutete Anpassung des Regelalgorithmus möglich, was das Fahrverhalten zudem robuster gegenüber Fehlwahrnehmungen der Fahrbahn machen.\\ \\
Mit der bereits erfolgreichen Teilnahme am Carolo-Cup 2015 im Rücken und dem weiteren dargelegten Entwicklungspotential steht das oTToCAR-Team dem nächsten Wettkampf im Jahr 2016 zuversichtlich gegenüber.