\section{Modellprädiktive Regelung}
Bei der modellprädiktiven Regelung (MPC) handelt es sich um eine Form der optimalen Regelung, bei der wiederholt eine Berechnung der optimalen Steuerung für ein System ausgehend von dessen aktuellem Zustand stattfindet. In vielen Bereichen finden MPCs immer häufiger Anwendung, da sie eine direkte Berücksichtigung von Beschränkungen erlauben und eine Form des strukturierten Reglerentwurfs ausgehend von der modellierten Systemdynamik darstellen. Dabei kann durch die geeignete Wahl der Kostenfunktion und deren Wichtungsparametern die Güte der des Reglers geziehlt beeinflusst werden. Allerdings ergeben sich auch Schwierigkeiten bei der Verwendung von MPCs. Zum einen ist die Konvergenz der Optimierung gegen einen optimalen Wert für die Optimierungsvariablen und die Stabiltät des geschlossenen Kreises insbesondere bei nichtlinearen Systemmodellen oft nur schwierig nachweisbar und zum anderen stellt das wiederholte Lösen des meist hochdimensionalen Optimierungsproblems während der Laufzeit in genügend schneller Geschwindigkeit eine große Herausforderung dar.
\\
\subsection{Problemformulierung für die modellprädiktive Regelung}
Im realen Anwenungsfall des oTToCAR-Projekts eignet sich eine Systemdarstellung in zeitdiskreter Form, bei der die Lösung des Optimierungsproblems weniger komplex ist und die ebenfalls zeitdiskreten Messwerte vom realen System weniger kompliziert integriert werden können. Demnach sind die diskretisierten Systemgleichungen wiefolgt gegeben:
\begin{align*}
  \boldsymbol{x}(k+1)&=\boldsymbol{f}\left ( \boldsymbol{x}(k), \boldsymbol{u}(k) \right )\\
  \boldsymbol{y}(k)&=\boldsymbol{g}\left ( \boldsymbol{x}(k) \right )\\
\end{align*}
mit den nichtlinearen Funktionen $\boldsymbol{f}\left ( \cdot \right )$ und $\boldsymbol{g}\left ( \cdot \right )$, wobei
\begin{align*}
  &\boldsymbol{x}(k) \in \mathcal{X}\subset\mathbb{R}^n\\
  &\boldsymbol{u}(k) \in \mathcal{U}\subset\mathbb{R}^m\\
  &\boldsymbol{y}(k) \in \mathcal{Y}\subset\mathbb{R}^r
\end{align*}
Ausgehend vom aktuellen Zustand $x(k)$ des zu regelnden Systems, der wenn nicht messbar geschätzt werden muss, wird anhand des Systemmodells das zukünftige Systemverhalten
\begin{align*}
  \boldsymbol{x_p}=\left\{ \boldsymbol{x}(k+1),\dots,\boldsymbol{x}(k+n_p)\right\}
\end{align*}
bis zum Prädiktionshorizont $n_p$ unter der Optimierung einer Sequenz von Eingängen
\begin{align*}
  \boldsymbol{u}=\left\{ \boldsymbol{u}(k),\dots,\boldsymbol{u}(k+n_c-1)\right\}
\end{align*}
bis zum Stellhorizont $n_c$ vorhergesagt. Aus der gefundenen optimalen Eingangssequenz $u^*$ wird der erste Eintrag $u^*(k)$ auf das zu regelnde System angewandt. Im nächsten Zeitschritt kann der neue Zustand gemessen bzw. geschätzt werden und die Optimierung beginnt von neuem. Ziel dabei ist es einer Referenztrajektorie $x_r$ zu folgen.\\ \\
Für das an jedem Zeitschritt $k$ zu lösende Minimierungsproblem wurde die benötigte Kostenfunktion $J$ in quadratische Form mit $x_p$ und $u$ als Optimierungsvariablen aufgestellt:
\begin{align*}
	\underset{\boldsymbol{x_p, u}}{\text{min}}\;J:=\sum_{i=k+1}^{k+n_p} \left [\boldsymbol{x}_{p}(i)-\boldsymbol{x}_{r}(i)\right ]^T\boldsymbol{Q}_i\left [\boldsymbol{x}_{p}(i)-\boldsymbol{x}_{r}(i)\right ] +\sum_{j=k}^{k+n_c-1} \boldsymbol{u}^T(i)\boldsymbol{R}_j\boldsymbol{u}\\
\end{align*}
Mit den Vektoren
\begin{align*}
	\boldsymbol{x}_p(k)&=\left [ \boldsymbol{x}_p(k+1\mid k),\dots,\boldsymbol{x}_p(k+n_p\mid k) \right ]^T\\
	\boldsymbol{x}_r(k)&=\left [ \boldsymbol{x}_r(k+1),\dots,\boldsymbol{x}_r(k+n_p) \right ]^T\\
	\boldsymbol{u}(k)&=\left [ \boldsymbol{u}(k),\dots,\boldsymbol{u}(k+n_c-1) \right ]^T
\end{align*}
und dazugehörigen Wichtungsmatrizen $\boldsymbol{Q}_i\quad(i=1, ...,n_p)$ und $\boldsymbol{R}_j\quad(j=0, ...,n_c-1)$, die folgende Vorausetzungen erfüllen müssen.\\
Linear MPC/Nonlinear MPC\\
Beschränkungen
\subsection{Konkrete Umsetzung der modellprädiktiven Regelung am oTToCAR}
u setzt sich zusammen aus Stellgröße Servo für den Lenkeinschlag und Stellgröße für den Motor für die Beschleunigung. Dabei wird n Schritte in die Zukunft geschaut, woraus ein Optimierungsvektor der Dimension R=irgendwas resultiert. Je größer man n wählt, desto besser lässt sich z.B. die Geschwindigkeit in Abhängigkeit zur Entfernung und Krümmung einer bevorstehenden Kurve ">anpassen"< oder Schwierigkeiten beim Durchfahren einer S-Kurve überwinden. Allerdings nimmt dadurch auch die Komplexität der Optimierung zu und man stößt schnell an die Grenzen der verfügbaren Rechenzeit/Recourcen.\\ \\
Verarbeitung der Polylinien\\
Interpolation\\
Parameter s aus der Geschwindigkeit bestimmen\\
An Polylinien entlang fahren\\ \\
Vereinfachungen\\
Aktuell wird die Motorsteuergröße nicht mitoptimiert und der Prädiktionshorizont beträgt 1. In dieser Konfiguration gibt es sicher Algorithmen wie LQR die Äquivalente Lösungen liefern und dabei besser nachweisbare Stabilität aufweisen. Es wurde sich trotzdem für die vereinfachte Variante entschieden, um diese in Zukunft so wie geplant zu erweitern.
Stabilität\\
Um Sicher zu gehen, dass der Algorithmus immer gegen ein Optimum konvergiert wurde die in implementierten Fall skalare Kostenfunktionen in der Parcour fahrt in möglichst vielen denkbaren Positionen aufgenommen und überprüft, dass sich kein Fall ergibt, in dem das Optimierungsproblem in relevanten Bereich nicht convex ist.
\\
Parabeln
\\
Es genügt den Scheitelpunkt der approximierten Variablen zu berechnen. genau genug.
Skalierbarkeit\\
Hinzufügen von Beschränkungen
Scheitelpunkt -> Newtonschritt der quadratischen Kostenfunktion
Aufwand\\
Nochmal überlegen\\
Ergebniss visualisieren: Plot der Punkte der Polylinie mit eingetragener Prädiktion des optimalen Lenkeinschlags\\
In der Vorbereitung auf den Wettkampf hat sich herausgestellt, dass die kurzen Testzeiten auf der Originalstrecke gut ausgenutzt werden muss. Dazu musste vorher extrahiert werden, welche Parameter entscheidenen Einfluss auf die Güte haben. Diese konnten dann online während der Fahrt mit dem dynamic reconfigure Paket getuned werden. (welche Parameter)\\
keine neue Polylinie\\
\subsection{Zukünftige Schritte}
Geschwindigkeit, mehrere Schritte in die Zukunft, Erstellung einer Karte mit Gewichtung der Confidence der Polylinien